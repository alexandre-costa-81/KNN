\documentclass[11pt]{beamer}
\usetheme{Warsaw}
\usepackage[utf8]{inputenc}
\usepackage[portuguese]{babel}
\usepackage[T1]{fontenc}
\usepackage{amsmath}
\usepackage{amsfonts}
\usepackage{amssymb}
\usepackage{graphicx}
\author{Leandro, Diego e Alexandre}
\title{k-Nearest Neighbors algorithm (KNN)}
%\setbeamercovered{transparent} 
%\setbeamertemplate{navigation symbols}{} 
%\logo{} 
%\institute{} 
%\date{} 
%\subject{} 
\begin{document}

\begin{frame}
\titlepage
\end{frame}

\begin{frame}
\tableofcontents
\end{frame}

\section{Introdução}
\begin{frame}{Objetivo}

\begin{itemize}
	\item Realizar um estudo sobre a aplicação do algoritmo KNN em FPGAs;
	\item Reconhecimento de indivíduos a partir dos movimentos e dados antropométricos \cite{granaidentificaccao};
	\item Foi realizado uma conexão entre o Kinect, a placa Altera Cyclone (2C35) e o computador para coleta em tempo real dos dados;
	\item Construido um ambiente experimental simulado. 
\end{itemize}

\end{frame}

\section{Conteúdo}
\begin{frame}{Kinect Versão 2}

\begin{itemize}
	\item O Kinect é equipado com:
	\begin{itemize}
		\item Câmera RGB;
		\item Sensor de profundidade composto de um emissor de luz infravermelho;
		\item Câmera sensível à profundidade. 
	\end{itemize}
	\item O sensor de profundidade do Kinect emite um padrão de infravermelho;
	\item A captura simultânea da imagem desse infravermelho com a câmera tradicional;
	\item Captura o infravermelho e bloqueia outras formas de onda;
	\item O processador de imagens do Kinect usa as posições relativas dos pontos no padrão do infravermelho para calcular o deslocamento da profundidade em cada posição de pixel na imagem.
\end{itemize}

\end{frame}

\begin{frame}{Kinect Versão 2}

\begin{itemize}
	\item Cada pixel do mapa de profundidade representa a distância cartesiana do plano da câmera até o objeto mais próximo, naquela coordenada (x, y) em particular;
	\item Se o valor do pixel é 0, isso indica que o sensor não encontrou nenhum objeto no seu espaço de alcance naquela localização (x, y).
	\item Essa projeção é mencionado como espaço de profundidade.
\end{itemize}

\end{frame}

\begin{frame}{Algoritmo KNN}

\begin{itemize}
	\item O algoritmo K-NN é utilizado para classificar um objeto não rotulado, baseado no rótulo de seus vizinhos mais próximos;
	\item Essa proximidade é baseada em uma métrica de distância entre dois pontos (distância Euclidiana);
	\item A regra de classificação do k-NN é associar a uma amostra de teste, o rótulo da maioria das categorias de seus “k” vizinhos mais próximos;
\end{itemize}

\end{frame}

\begin{frame}{Custo}

\begin{itemize}
	\item Custo
\end{itemize}

\end{frame}

\section{Conclusão}
\begin{frame}{Conclusão}

\begin{itemize}
	\item Conclusão
\end{itemize}

\end{frame}

\section{Referências}
\begin{frame}{Referências}
  
	\bibliography{ref}{}
	\bibliographystyle{plain}

\end{frame}

\end{document}