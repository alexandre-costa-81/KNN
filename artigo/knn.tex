\documentclass[12pt]{article}

\usepackage{sbc-template}

\usepackage{graphicx,url}

\usepackage[brazil]{babel}   
%\usepackage[latin1]{inputenc}  

     
\sloppy

\title{Implementa\c{c}\~ao do k-nearest neighbors -- KNN\\ em VHDL}

\author{Alexandre G. da Costa\inst{1}, Diego P. Jaccottet\inst{1}, Leandro W.
  Dias\inst{1} }


\address{CDTec -- Universidade Federal de Pelotas
  (UFPEL)\\
  CEP 96.010-610 -- Pelotas -- RS -- Brazil
  \email{alexandre.costa@inf.ufpel.edu.br, \{diego.porto.j,lwdias\}@gmail.com}
}

\begin{document} 

\maketitle

\begin{abstract}
  This paper describes ...
\end{abstract}
     
\begin{resumo} 
  Este artigo descreve ...
\end{resumo}


\section{Introdu\c{c}\~ao}

O objetivo deste projeto é implementar o algoritmo KNN na placa Altera Cyclone
(2C35) FPGA de 35.000 LEs, afim de reconhecer indivíduos. Sendo que a entrada
será o esqueleto do indivíduo, esse esqueleto é inferido a partir de filmagem
tridimensional realizada pelo Kinect Versão 2 pelo seu SDK. 

\section{Kinect Vers\~ao 2} \label{sec:kinectversion2}

O Kinect é um sensor equipado de uma câmera RGB, um sensor de profundidade
composto de um emissor de luz infravermelho e uma câmera sensível à
profundidade. 

O principio básico por trás do sensor de profundidade do Kinect é a emissão
de um padrão de infravermelho e a captura simultânea da imagem desse
infravermelho com uma câmera tradicional equipada com um filtro, que permite
capturar o infravermelho e bloquear outras formas de onda. O processador de
imagens do Kinect usa as posições relativas dos pontos no padrão do
infravermelho para calcular o deslocamento da profundidade em cada posição de
pixel na imagem.

Cada pixel do mapa de profundidade representa a distância cartesiana,
em milímetros, do plano da câmera até o objeto mais próximo, naquela coordenada
(x, y) em particular. Se o valor do pixel é 0, isso indica que o sensor não
encontrou nenhum objeto no seu espaço de alcance naquela localização (x, y).
Essa projeção é mencionado como espaço de profundidade. Os valores correntes
de profundidade são as distâncias do plano da câmera, ao invés das do sensor
propriamente dito.

\section{Algoritmo KNN}

O algoritmo de aprendizado supervisionado KNN é utilizado para classificar um
objeto não rotulado, baseado no rótulo de seus vizinhos mais próximos em um
espaço de exemplos. Essa proximidade é, frequentemente, baseada em uma métrica
de distância entre dois pontos, por exemplo, a distância Euclidiana. De maneira
simplificada, a regra de classificação do KNN é associar a uma amostra de
teste, o rótulo da maioria das categorias de seus ``k'' vizinhos mais próximos. 

Um conjunto de treino X consiste em n pares de vetores e rótulos, dispersos em
um espaço de classes. Dado um novo par (x; θ), onde apenas a medida x é
observável, o valor de θ é estimado pela utilização dos dados contidos no
conjunto X com os vetores e rótulos já conhecidos (supervisionado). Um vetor x'
é um vizinho mais próximo de x, se segundo a Equação:


\section{Descric\~ao VHDL do Kinect vers\~ao 2}

Esta seção descreve o Kinect versão 2 em VHDL...

\section{Descric\~ao VHDL do algoritmo KNN}

Esta seção descreve o algoritmo KNN em VHDL


\section{Conclus\~ao}\label{sec:figs}

Este trabalho desenvolveu um algoritmo em VHDL que...

%(Figure~\ref{fig:exampleFig1}), otherwise justified and indented by 0.8cm on
%both margins, as shown in Figure~\ref{fig:exampleFig2}. The caption font must

%\begin{figure}[ht]
%\centering
%\includegraphics[width=.5\textwidth]{fig1.jpg}
%\caption{A typical figure}
%\label{fig:exampleFig1}
%\end{figure}

%\begin{figure}[ht]
%\centering
%\includegraphics[width=.3\textwidth]{fig2.jpg}
%\caption{This figure is an example of a figure caption taking more than one
%  line and justified considering margins mentioned in Section~\ref{sec:figs}.}
%\label{fig:exampleFig2}
%\end{figure}

%\begin{table}[ht]
%\centering
%\caption{Variables to be considered on the evaluation of interaction
%  techniques}
%\label{tab:exTable1}
%\includegraphics[width=.7\textwidth]{table.jpg}
%\end{table}

%Bibliographic references must be unambiguous and uniform.  We recommend giving
%the author names references in brackets, e.g. \cite{knuth:84},
%\cite{boulic:91}, and \cite{smith:99}.

%The references must be listed using 12 point font size, with 6 points of space
%before each reference. The first line of each reference should not be
%indented, while the subsequent should be indented by 0.5 cm.

\bibliographystyle{sbc}
\bibliography{sbc-template}

\end{document}
