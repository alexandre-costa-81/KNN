\section{Introdu\c{c}\~ao}

FPGAS são muito usadas para protótipos, onde não se justifica um projeto ASIC
personalizado. Entre as vantagens das FPGAS estão o alto paralelismo e
adaptabilidade, com paralelismo para dados, tarefas, pipeline ou uma mistura
desses. FPGAS também são muito eficientes na questão da quantidade de bits
necessários para uma tarefa. Em FPGAS a lógica de instruções e de-codificação
de endereços se torna desnecessária, isso possibilita alcançar throughput muito
maior que em processadores tradicionais, assim como uma redução da quantidade
de energia usada. FPGAS vem se tornando cada vez mais poderosas, o que torna
seu uso cada vez mais interessante \cite{najjar2003high}.

O uso de gestos é um dos meios de Interação Humano-Computador (Human-Computer-
Interaction - HCI) mais naturais e intuitivos. Além disso existem aplicações em
realidade virtual, linguagem de sinais, e jogos de computador. Criar sistemas
para o reconhecimento de gestos ainda é um problema desafiador. Métodos usando
imagens de profundidade vem se tornando cada vez mais populares, pois é mais
robusto ao fundos de imagem confusos \cite{ren2011robust}. O uso do Kinect pode
simplificar muitas das etapas no processamento de vídeos, já que ele captura e
pré-processa as imagens \cite{Andersson:2014}.

O objetivo deste projeto é realizar uma interface entre o sensor Kinect Versão 2,
a placa Altera Cyclone (2C35) e o computador para coleta em tempo real dos dados para
executar em um algoritimo de classificação com o objetivo de reconhecimento de classes.